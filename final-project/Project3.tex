\documentclass[11pt]{article}

\usepackage{graphicx}
\usepackage{lscape}
\usepackage{amsmath}
\usepackage{mathrsfs}
\usepackage[pdftex]{color}
\usepackage{textcomp}
\usepackage{hyperref}

\setlength{\parindent}{0in}

\hypersetup{
    bookmarks=true,                       % show bookmarks bar?
    unicode=false,                        % non-Latin characters in Acrobat’s bookmarks
    pdftoolbar=true,                      % show Acrobat’s toolbar?
    pdfmenubar=true,                      % show Acrobat’s menu?
    pdffitwindow=false,                   % window fit to page when opened
    pdfstartview={FitH},                  % fits the width of the page to the window
    pdftitle={My title},                  % title
    pdfauthor={Author},                   % author
    pdfsubject={Subject},                 % subject of the document
    pdfcreator={Creator},                 % creator of the document
    pdfproducer={Producer},               % producer of the document
    pdfkeywords={keyword1} {key2} {key3}, % list of keywords
    pdfnewwindow=true,                    % links in new window
    colorlinks=true,                      % false: boxed links; true: colored links
    linkcolor=red,                        % color of internal links
    citecolor=green,                      % color of links to bibliography
    filecolor=magenta,                    % color of file links
    urlcolor=blue                         % color of external links
}

\begin{document}\pagestyle{empty}

\textbf{Finance 6320, Spring 2014} \\
\textbf{Project 3 Description}     \\

\bigskip
This example is taken from Chapter 4 of the book \textit{Implementing Derivatives Models} by Clewlow 
and Strickland. In this project you will price a European fixed strike lookback call option. This 
option pays the difference, if positive, between the maximum of a set of observations (called fixings)
of the asset price $S_{t_{i}}$ at dates $t_{i}; i = 1, \ldots, N$ and the strike price. Thus the
payoff at the maturity date is 

\medskip
\begin{equation*}
\max{(0, \max{(S_{t_{i}}; i = 1, \ldots, N)} - K)}
\end{equation*}

\bigskip
We will also assume that the asset price and the variance of the asset price returns $V = \sigma^{2}$
are governed by the following stochastic differential equations:

\medskip
\begin{eqnarray*}
dS &=& r S dt + \sigma S dz_{1} \\
dV &=& \alpha (\bar{V} - V) dt + \xi \sqrt{V} dz_{2}
\end{eqnarray*}

\medskip
and that the Wiener processes $dz_{1}$ and $dz_{2}$ are uncorrelated, though this can easily be
generalized. 

\bigskip
There is no analytical solution for the price of a European fixed strike lookback call option
with discrete fixings and stochastic volatility. However, there is a simple analytical formula
for the price of a continuous fixing fixed strike lookback call with constant volatility.

\begin{multline*}
C_{FSLBCall} = G + Se^{\delta T} N(x + \sigma \sqrt{T}) - Ke^{-rT}N(x) \\
                                  -\frac{S}{B} \left( e^{-rT} \left(\frac{E}{S} \right)^{B} N(x + (1 - B) \sigma \sqrt{T}) 
                                 -e^{-\delta T} N(x + \sigma \sqrt{T}) \right)
\end{multline*}    

\newpage
where

\medskip
\begin{equation*}
\begin{cases} \mbox{if} \quad K \ge M, & \, \mbox{then} \quad E = K, G = 0 \\
              \mbox{if} \quad K <   M, & \, \mbox{then} \quad E = M, G = e^{-rT}(M - K) \end{cases}
\end{equation*}

\medskip
and

\medskip
\begin{eqnarray*}
B &=& \frac{2(r - \delta)}{\sigma^{2}} \\
x &=& \frac{\ln{\left(\frac{S}{E}\right)} + \left((r-\delta) - \frac{1}{2}\sigma^{2} \right)T}{\sigma \sqrt{T}}
\end{eqnarray*}

\medskip
and $M$ is the current known maximum. We can therefore use the continuously fixed floating strike lookback
call option formula to compute \textit{delta}, \textit{gamma}, and \textit{vega} hedge control variates.
Rather than differentiate the above equation with respect to the asset price twice and volatility once
which would lead to extremely complex expressions it is more efficient to use finite difference approximations
to the partial differentials for \textit{gamma} and \textit{vega}.

\bigskip
\bigskip
A measure of the simulation error is the standard deviation of $\hat{C_{0}}$ which is called the standard error $SE(\: \cdot \:)$
and can be estimated as the sample standard deviation of $C_{0,j}$ divided by the square root of the number of
samples.

\medskip
\begin{equation*}
SE(\hat{C_{0}}) = \frac{SD(C_{0,j})}{\sqrt{M}}
\end{equation*}

\medskip
where

\medskip
\begin{equation*}
SD(C_{0,j}) = \sqrt{\frac{1}{M-1} \sum\limits_{j=1}^{M} (C_{0,j} - \hat{C_{0}})^{2}}
\end{equation*}

\newpage
The deliverable for this project, in addittion to the project source code, is a table that presents the results
of the simulation.  For example, fill in the missing data for the following table:



\bigskip
\begin{table}[h!]
\centering
\begin{tabular}{lc|c|c}
\hline
\hline
                              &                            & Standard & Computation \\
                              & Price \phantom{            } & Error    & Time        \\
                               \hline
Simple estimate               & &          &                      \\
Antithetic variate       & &          &                      \\
Control variates         & &          &                      \\
Combined variates             & &          &                      \\                               
                                                              
\hline
\hline
\end{tabular}
\end{table}

\bigskip
\bigskip
The parameters for the problem are given in the following table.

\bigskip
\begin{table}[h!]
\centering
\begin{tabular}{lr}
\hline
\hline
Strike Price                   &    100    \\ 
Time to Maturity               &    1 year \\
Initial asset price            &    100    \\
Volatility                     &    $20\%$ \\
Risk--free rate                &    $6\%$  \\
Continuous Dividend Yield      &    $3\%$  \\
Mean reversion rate ($\alpha$) &    5.0    \\
Volatility of Volatility ($\xi$) &  0.02   \\
Number of time steps           &    52     \\
Number of simulations          &    1000   \\                                                              
\hline
\hline
\end{tabular}
\end{table}

\end{document}


